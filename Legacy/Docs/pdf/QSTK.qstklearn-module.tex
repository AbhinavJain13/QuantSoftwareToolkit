%
% API Documentation for QSTK
% Package QSTK.qstklearn
%
% Generated by epydoc 3.0.1
% [Mon Mar  5 00:49:20 2012]
%

%%%%%%%%%%%%%%%%%%%%%%%%%%%%%%%%%%%%%%%%%%%%%%%%%%%%%%%%%%%%%%%%%%%%%%%%%%%
%%                          Module Description                           %%
%%%%%%%%%%%%%%%%%%%%%%%%%%%%%%%%%%%%%%%%%%%%%%%%%%%%%%%%%%%%%%%%%%%%%%%%%%%

    \index{QSTK \textit{(package)}!QSTK.qstklearn \textit{(package)}|(}
\section{Package QSTK.qstklearn}

    \label{QSTK:qstklearn}

%%%%%%%%%%%%%%%%%%%%%%%%%%%%%%%%%%%%%%%%%%%%%%%%%%%%%%%%%%%%%%%%%%%%%%%%%%%
%%                                Modules                                %%
%%%%%%%%%%%%%%%%%%%%%%%%%%%%%%%%%%%%%%%%%%%%%%%%%%%%%%%%%%%%%%%%%%%%%%%%%%%

\subsection{Modules}

\begin{itemize}
\setlength{\parskip}{0ex}
\item \textbf{fastknn}: This package is an implementation of a novel improvement to KNN which 
speeds up query times



  \textit{(Section \ref{QSTK:qstklearn:fastknn}, p.~\pageref{QSTK:qstklearn:fastknn})}

\item \textbf{gendata}
  \textit{(Section \ref{QSTK:qstklearn:gendata}, p.~\pageref{QSTK:qstklearn:gendata})}

\item \textbf{hmm}: This package includes code for representing and learning HMM's.



  \textit{(Section \ref{QSTK:qstklearn:hmm}, p.~\pageref{QSTK:qstklearn:hmm})}

\item \textbf{kdtknn}: A simple wrapper for scipy.spatial.kdtree.KDTree for doing KNN



  \textit{(Section \ref{QSTK:qstklearn:kdtknn}, p.~\pageref{QSTK:qstklearn:kdtknn})}

\item \textbf{mldiagnostics}
  \textit{(Section \ref{QSTK:qstklearn:mldiagnostics}, p.~\pageref{QSTK:qstklearn:mldiagnostics})}

\item \textbf{parallelknn}: This is an implementation of the K nearest neighbor learning algorithm. The
implementation is trivial in that the near neighbors are calculated 
naively- without any smart tricks. Euclidean distance is used to calculate 
the distance between two points. The implementation also provides some 
coarse parallelism. If the par\_query function is used then the query 
points are split up equally amongst threads and their near neighbors are 
calculated in parallel. If the number of threads to use is not specified 
then no of threads  = no of cores as returned by the cpu\_count function. 
This may not be ideal.



  \textit{(Section \ref{QSTK:qstklearn:parallelknn}, p.~\pageref{QSTK:qstklearn:parallelknn})}

\end{itemize}


%%%%%%%%%%%%%%%%%%%%%%%%%%%%%%%%%%%%%%%%%%%%%%%%%%%%%%%%%%%%%%%%%%%%%%%%%%%
%%                               Variables                               %%
%%%%%%%%%%%%%%%%%%%%%%%%%%%%%%%%%%%%%%%%%%%%%%%%%%%%%%%%%%%%%%%%%%%%%%%%%%%

  \subsection{Variables}

    \vspace{-1cm}
\hspace{\varindent}\begin{longtable}{|p{\varnamewidth}|p{\vardescrwidth}|l}
\cline{1-2}
\cline{1-2} \centering \textbf{Name} & \centering \textbf{Description}& \\
\cline{1-2}
\endhead\cline{1-2}\multicolumn{3}{r}{\small\textit{continued on next page}}\\\endfoot\cline{1-2}
\endlastfoot\raggedright \_\-\_\-p\-a\-c\-k\-a\-g\-e\-\_\-\_\- & \raggedright \textbf{Value:} 
{\tt None}&\\
\cline{1-2}
\end{longtable}

    \index{QSTK \textit{(package)}!QSTK.qstklearn \textit{(package)}|)}
