%
% API Documentation for QSTK
% Module QSTK.qstklearn.kdtknn
%
% Generated by epydoc 3.0.1
% [Mon Mar  5 00:49:20 2012]
%

%%%%%%%%%%%%%%%%%%%%%%%%%%%%%%%%%%%%%%%%%%%%%%%%%%%%%%%%%%%%%%%%%%%%%%%%%%%
%%                          Module Description                           %%
%%%%%%%%%%%%%%%%%%%%%%%%%%%%%%%%%%%%%%%%%%%%%%%%%%%%%%%%%%%%%%%%%%%%%%%%%%%

    \index{QSTK \textit{(package)}!QSTK.qstklearn \textit{(package)}!QSTK.qstklearn.kdtknn \textit{(module)}|(}
\section{Module QSTK.qstklearn.kdtknn}

    \label{QSTK:qstklearn:kdtknn}
A simple wrapper for scipy.spatial.kdtree.KDTree for doing KNN


%%%%%%%%%%%%%%%%%%%%%%%%%%%%%%%%%%%%%%%%%%%%%%%%%%%%%%%%%%%%%%%%%%%%%%%%%%%
%%                               Functions                               %%
%%%%%%%%%%%%%%%%%%%%%%%%%%%%%%%%%%%%%%%%%%%%%%%%%%%%%%%%%%%%%%%%%%%%%%%%%%%

  \subsection{Functions}

    \label{QSTK:qstklearn:kdtknn:getflatcsv}
    \index{QSTK \textit{(package)}!QSTK.qstklearn \textit{(package)}!QSTK.qstklearn.kdtknn \textit{(module)}!QSTK.qstklearn.kdtknn.getflatcsv \textit{(function)}}

    \vspace{0.5ex}

\hspace{.8\funcindent}\begin{boxedminipage}{\funcwidth}

    \raggedright \textbf{getflatcsv}(\textit{fname})

\setlength{\parskip}{2ex}
\setlength{\parskip}{1ex}
    \end{boxedminipage}

    \label{QSTK:qstklearn:kdtknn:testgendata}
    \index{QSTK \textit{(package)}!QSTK.qstklearn \textit{(package)}!QSTK.qstklearn.kdtknn \textit{(module)}!QSTK.qstklearn.kdtknn.testgendata \textit{(function)}}

    \vspace{0.5ex}

\hspace{.8\funcindent}\begin{boxedminipage}{\funcwidth}

    \raggedright \textbf{testgendata}()

\setlength{\parskip}{2ex}
\setlength{\parskip}{1ex}
    \end{boxedminipage}

    \label{QSTK:qstklearn:kdtknn:test}
    \index{QSTK \textit{(package)}!QSTK.qstklearn \textit{(package)}!QSTK.qstklearn.kdtknn \textit{(module)}!QSTK.qstklearn.kdtknn.test \textit{(function)}}

    \vspace{0.5ex}

\hspace{.8\funcindent}\begin{boxedminipage}{\funcwidth}

    \raggedright \textbf{test}()

\setlength{\parskip}{2ex}
\setlength{\parskip}{1ex}
    \end{boxedminipage}


%%%%%%%%%%%%%%%%%%%%%%%%%%%%%%%%%%%%%%%%%%%%%%%%%%%%%%%%%%%%%%%%%%%%%%%%%%%
%%                               Variables                               %%
%%%%%%%%%%%%%%%%%%%%%%%%%%%%%%%%%%%%%%%%%%%%%%%%%%%%%%%%%%%%%%%%%%%%%%%%%%%

  \subsection{Variables}

    \vspace{-1cm}
\hspace{\varindent}\begin{longtable}{|p{\varnamewidth}|p{\vardescrwidth}|l}
\cline{1-2}
\cline{1-2} \centering \textbf{Name} & \centering \textbf{Description}& \\
\cline{1-2}
\endhead\cline{1-2}\multicolumn{3}{r}{\small\textit{continued on next page}}\\\endfoot\cline{1-2}
\endlastfoot\raggedright \_\-\_\-p\-a\-c\-k\-a\-g\-e\-\_\-\_\- & \raggedright \textbf{Value:} 
{\tt \texttt{'}\texttt{QSTK.qstklearn}\texttt{'}}&\\
\cline{1-2}
\end{longtable}


%%%%%%%%%%%%%%%%%%%%%%%%%%%%%%%%%%%%%%%%%%%%%%%%%%%%%%%%%%%%%%%%%%%%%%%%%%%
%%                           Class Description                           %%
%%%%%%%%%%%%%%%%%%%%%%%%%%%%%%%%%%%%%%%%%%%%%%%%%%%%%%%%%%%%%%%%%%%%%%%%%%%

    \index{QSTK \textit{(package)}!QSTK.qstklearn \textit{(package)}!QSTK.qstklearn.kdtknn \textit{(module)}!QSTK.qstklearn.kdtknn.kdtknn \textit{(class)}|(}
\subsection{Class kdtknn}

    \label{QSTK:qstklearn:kdtknn:kdtknn}
\begin{tabular}{cccccc}
% Line for object, linespec=[False]
\multicolumn{2}{r}{\settowidth{\BCL}{object}\multirow{2}{\BCL}{object}}
&&
  \\\cline{3-3}
  &&\multicolumn{1}{c|}{}
&&
  \\
&&\multicolumn{2}{l}{\textbf{QSTK.qstklearn.kdtknn.kdtknn}}
\end{tabular}

A simple wrapper of scipy.spatial.kdtree.KDTree

Since the scipy KDTree implementation does not allow for incrementally 
adding data points, the entire KD-tree is rebuilt on the first call to 
'query' after a call to 'addEvidence'. For this reason it is more efficient
to add training data in batches.


%%%%%%%%%%%%%%%%%%%%%%%%%%%%%%%%%%%%%%%%%%%%%%%%%%%%%%%%%%%%%%%%%%%%%%%%%%%
%%                                Methods                                %%
%%%%%%%%%%%%%%%%%%%%%%%%%%%%%%%%%%%%%%%%%%%%%%%%%%%%%%%%%%%%%%%%%%%%%%%%%%%

  \subsubsection{Methods}

    \vspace{0.5ex}

\hspace{.8\funcindent}\begin{boxedminipage}{\funcwidth}

    \raggedright \textbf{\_\_init\_\_}(\textit{self}, \textit{k}={\tt 3}, \textit{method}={\tt \texttt{'}\texttt{mean}\texttt{'}}, \textit{leafsize}={\tt 10})

    \vspace{-1.5ex}

    \rule{\textwidth}{0.5\fboxrule}
\setlength{\parskip}{2ex}
    Basic setup.

\setlength{\parskip}{1ex}
      Overrides: object.\_\_init\_\_

    \end{boxedminipage}

    \label{QSTK:qstklearn:kdtknn:kdtknn:addEvidence}
    \index{QSTK \textit{(package)}!QSTK.qstklearn \textit{(package)}!QSTK.qstklearn.kdtknn \textit{(module)}!QSTK.qstklearn.kdtknn.kdtknn \textit{(class)}!QSTK.qstklearn.kdtknn.kdtknn.addEvidence \textit{(method)}}

    \vspace{0.5ex}

\hspace{.8\funcindent}\begin{boxedminipage}{\funcwidth}

    \raggedright \textbf{addEvidence}(\textit{self}, \textit{dataX}, \textit{dataY}={\tt None})

    \vspace{-1.5ex}

    \rule{\textwidth}{0.5\fboxrule}
\setlength{\parskip}{2ex}
\setlength{\parskip}{1ex}
      \textbf{Parameters}
      \vspace{-1ex}

      \begin{quote}
        \begin{Ventry}{xxxxx}

          \item[dataX]

          Data to add, either entire set with classification as last 
          column, or not if the Y data is provided explicitly.  Must be 
          same width as previously appended data.

          \item[dataY]

          Optional, can be used

          'data' should be a numpy array matching the same dimensions as 
          any data provided in previous calls to addEvidence, with dataY as
          the training label.

        \end{Ventry}

      \end{quote}

    \end{boxedminipage}

    \label{QSTK:qstklearn:kdtknn:kdtknn:rebuildKDT}
    \index{QSTK \textit{(package)}!QSTK.qstklearn \textit{(package)}!QSTK.qstklearn.kdtknn \textit{(module)}!QSTK.qstklearn.kdtknn.kdtknn \textit{(class)}!QSTK.qstklearn.kdtknn.kdtknn.rebuildKDT \textit{(method)}}

    \vspace{0.5ex}

\hspace{.8\funcindent}\begin{boxedminipage}{\funcwidth}

    \raggedright \textbf{rebuildKDT}(\textit{self})

    \vspace{-1.5ex}

    \rule{\textwidth}{0.5\fboxrule}
\setlength{\parskip}{2ex}
    Force the internal KDTree to be rebuilt.

\setlength{\parskip}{1ex}
    \end{boxedminipage}

    \label{QSTK:qstklearn:kdtknn:kdtknn:query}
    \index{QSTK \textit{(package)}!QSTK.qstklearn \textit{(package)}!QSTK.qstklearn.kdtknn \textit{(module)}!QSTK.qstklearn.kdtknn.kdtknn \textit{(class)}!QSTK.qstklearn.kdtknn.kdtknn.query \textit{(method)}}

    \vspace{0.5ex}

\hspace{.8\funcindent}\begin{boxedminipage}{\funcwidth}

    \raggedright \textbf{query}(\textit{self}, \textit{points}, \textit{k}={\tt None}, \textit{method}={\tt None})

    \vspace{-1.5ex}

    \rule{\textwidth}{0.5\fboxrule}
\setlength{\parskip}{2ex}
    Classify a set of test points given their k nearest neighbors.

    'points' should be a numpy array with each row corresponding to a 
    specific query. Returns the estimated class according to supplied 
    method (currently only 'mode' and 'mean' are supported)

\setlength{\parskip}{1ex}
    \end{boxedminipage}


\large{\textbf{\textit{Inherited from object}}}

\begin{quote}
\_\_delattr\_\_(), \_\_format\_\_(), \_\_getattribute\_\_(), \_\_hash\_\_(), \_\_new\_\_(), \_\_reduce\_\_(), \_\_reduce\_ex\_\_(), \_\_repr\_\_(), \_\_setattr\_\_(), \_\_sizeof\_\_(), \_\_str\_\_(), \_\_subclasshook\_\_()
\end{quote}

%%%%%%%%%%%%%%%%%%%%%%%%%%%%%%%%%%%%%%%%%%%%%%%%%%%%%%%%%%%%%%%%%%%%%%%%%%%
%%                              Properties                               %%
%%%%%%%%%%%%%%%%%%%%%%%%%%%%%%%%%%%%%%%%%%%%%%%%%%%%%%%%%%%%%%%%%%%%%%%%%%%

  \subsubsection{Properties}

    \vspace{-1cm}
\hspace{\varindent}\begin{longtable}{|p{\varnamewidth}|p{\vardescrwidth}|l}
\cline{1-2}
\cline{1-2} \centering \textbf{Name} & \centering \textbf{Description}& \\
\cline{1-2}
\endhead\cline{1-2}\multicolumn{3}{r}{\small\textit{continued on next page}}\\\endfoot\cline{1-2}
\endlastfoot\multicolumn{2}{|l|}{\textit{Inherited from object}}\\
\multicolumn{2}{|p{\varwidth}|}{\raggedright \_\_class\_\_}\\
\cline{1-2}
\end{longtable}

    \index{QSTK \textit{(package)}!QSTK.qstklearn \textit{(package)}!QSTK.qstklearn.kdtknn \textit{(module)}!QSTK.qstklearn.kdtknn.kdtknn \textit{(class)}|)}
    \index{QSTK \textit{(package)}!QSTK.qstklearn \textit{(package)}!QSTK.qstklearn.kdtknn \textit{(module)}|)}
