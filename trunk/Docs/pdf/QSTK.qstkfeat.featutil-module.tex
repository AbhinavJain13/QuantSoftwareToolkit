%
% API Documentation for QSTK
% Module QSTK.qstkfeat.featutil
%
% Generated by epydoc 3.0.1
% [Mon Mar  5 00:49:20 2012]
%

%%%%%%%%%%%%%%%%%%%%%%%%%%%%%%%%%%%%%%%%%%%%%%%%%%%%%%%%%%%%%%%%%%%%%%%%%%%
%%                          Module Description                           %%
%%%%%%%%%%%%%%%%%%%%%%%%%%%%%%%%%%%%%%%%%%%%%%%%%%%%%%%%%%%%%%%%%%%%%%%%%%%

    \index{QSTK \textit{(package)}!QSTK.qstkfeat \textit{(package)}!QSTK.qstkfeat.featutil \textit{(module)}|(}
\section{Module QSTK.qstkfeat.featutil}

    \label{QSTK:qstkfeat:featutil}
(c) 2011, 2012 Georgia Tech Research Corporation This source code is 
released under the New BSD license.  Please see 
http://wiki.quantsoftware.org/index.php?title=QSTK\_License for license 
details.

Created on Nov 7, 2011

\textbf{Author:} John Cornwell



\textbf{Contact:} JohnWCornwellV@gmail.com




%%%%%%%%%%%%%%%%%%%%%%%%%%%%%%%%%%%%%%%%%%%%%%%%%%%%%%%%%%%%%%%%%%%%%%%%%%%
%%                               Functions                               %%
%%%%%%%%%%%%%%%%%%%%%%%%%%%%%%%%%%%%%%%%%%%%%%%%%%%%%%%%%%%%%%%%%%%%%%%%%%%

  \subsection{Functions}

    \label{QSTK:qstkfeat:featutil:getMarketRel}
    \index{QSTK \textit{(package)}!QSTK.qstkfeat \textit{(package)}!QSTK.qstkfeat.featutil \textit{(module)}!QSTK.qstkfeat.featutil.getMarketRel \textit{(function)}}

    \vspace{0.5ex}

\hspace{.8\funcindent}\begin{boxedminipage}{\funcwidth}

    \raggedright \textbf{getMarketRel}(\textit{dData}, \textit{sRel}={\tt \texttt{'}\texttt{\$SPX}\texttt{'}})

    \vspace{-1.5ex}

    \rule{\textwidth}{0.5\fboxrule}
\setlength{\parskip}{2ex}
\setlength{\parskip}{1ex}
      \textbf{Parameters}
      \vspace{-1ex}

      \begin{quote}
        \begin{Ventry}{xxxxxxxxxx}

          \item[dData, -, Dictionary, containing, data, to, be, used, requires, specific, naming]

          open/high/low/close/volume @param sRel - Stock ticker to make the
          data relative to, \$SPX is default.

        \end{Ventry}

      \end{quote}

      \textbf{Return Value}
    \vspace{-1ex}

      \begin{quote}
      Dictionary of market relative values

      \end{quote}

    \end{boxedminipage}

    \label{QSTK:qstkfeat:featutil:applyFeatures}
    \index{QSTK \textit{(package)}!QSTK.qstkfeat \textit{(package)}!QSTK.qstkfeat.featutil \textit{(module)}!QSTK.qstkfeat.featutil.applyFeatures \textit{(function)}}

    \vspace{0.5ex}

\hspace{.8\funcindent}\begin{boxedminipage}{\funcwidth}

    \raggedright \textbf{applyFeatures}(\textit{dData}, \textit{lfcFeatures}, \textit{ldArgs}, \textit{sMarketRel}={\tt None}, \textit{sLog}={\tt None})

    \vspace{-1.5ex}

    \rule{\textwidth}{0.5\fboxrule}
\setlength{\parskip}{2ex}
\setlength{\parskip}{1ex}
      \textbf{Parameters}
      \vspace{-1ex}

      \begin{quote}
        \begin{Ventry}{xxxxxxxxxxx}

          \item[dData, -, Dictionary, containing, data, to, be, used, requires, specific, naming]

          open/high/low/close/volume

          \item[lfcFeatures]

          List of feature functions, most likely coming from features.py

          \item[ldArgs]

          List of dictionaries containing arguments, passed as **kwargs 
          There is a special argument 'MR', if it exists, the data will be 
          made market relative

          \item[sMarketRel]

          If not none, the data will all be made relative to the symbol 
          provided

          \item[sLog]

          If not None, will be filename to log all of the features to

        \end{Ventry}

      \end{quote}

      \textbf{Return Value}
    \vspace{-1ex}

      \begin{quote}
      list of dataframes containing values

      \end{quote}

    \end{boxedminipage}

    \label{QSTK:qstkfeat:featutil:loadFeatures}
    \index{QSTK \textit{(package)}!QSTK.qstkfeat \textit{(package)}!QSTK.qstkfeat.featutil \textit{(module)}!QSTK.qstkfeat.featutil.loadFeatures \textit{(function)}}

    \vspace{0.5ex}

\hspace{.8\funcindent}\begin{boxedminipage}{\funcwidth}

    \raggedright \textbf{loadFeatures}(\textit{sLog})

    \vspace{-1.5ex}

    \rule{\textwidth}{0.5\fboxrule}
\setlength{\parskip}{2ex}
\setlength{\parskip}{1ex}
      \textbf{Parameters}
      \vspace{-1ex}

      \begin{quote}
        \begin{Ventry}{xxxx}

          \item[sLog]

          Filename of features.

        \end{Ventry}

      \end{quote}

      \textbf{Return Value}
    \vspace{-1ex}

      \begin{quote}
      Numpy array containing values

      \end{quote}

    \end{boxedminipage}

    \label{QSTK:qstkfeat:featutil:stackSyms}
    \index{QSTK \textit{(package)}!QSTK.qstkfeat \textit{(package)}!QSTK.qstkfeat.featutil \textit{(module)}!QSTK.qstkfeat.featutil.stackSyms \textit{(function)}}

    \vspace{0.5ex}

\hspace{.8\funcindent}\begin{boxedminipage}{\funcwidth}

    \raggedright \textbf{stackSyms}(\textit{ldfFeatures}, \textit{dtStart}={\tt None}, \textit{dtEnd}={\tt None}, \textit{lsSym}={\tt None}, \textit{bDelNan}={\tt True}, \textit{bShowRemoved}={\tt False})

    \vspace{-1.5ex}

    \rule{\textwidth}{0.5\fboxrule}
\setlength{\parskip}{2ex}
\setlength{\parskip}{1ex}
      \textbf{Parameters}
      \vspace{-1ex}

      \begin{quote}
        \begin{Ventry}{xxxxxxxxxxx}

          \item[ldfFeatures]

          List of data frames of features.

          \item[dtStart]

          Start time, if None, uses all

          \item[dtEnd]

          End time, if None uses all

          \item[lsSym]

          List of symbols to use, if None, all are used.

          \item[bDelNan]

          Optional, default is true: delete all rows with a NaN in it

        \end{Ventry}

      \end{quote}

      \textbf{Return Value}
    \vspace{-1ex}

      \begin{quote}
      Numpy array containing all features as columns and all

      \end{quote}

    \end{boxedminipage}

    \label{QSTK:qstkfeat:featutil:normFeatures}
    \index{QSTK \textit{(package)}!QSTK.qstkfeat \textit{(package)}!QSTK.qstkfeat.featutil \textit{(module)}!QSTK.qstkfeat.featutil.normFeatures \textit{(function)}}

    \vspace{0.5ex}

\hspace{.8\funcindent}\begin{boxedminipage}{\funcwidth}

    \raggedright \textbf{normFeatures}(\textit{naFeatures}, \textit{fMin}, \textit{fMax}, \textit{bAbsolute}, \textit{bIgnoreLast}={\tt True})

    \vspace{-1.5ex}

    \rule{\textwidth}{0.5\fboxrule}
\setlength{\parskip}{2ex}
\setlength{\parskip}{1ex}
      \textbf{Parameters}
      \vspace{-1ex}

      \begin{quote}
        \begin{Ventry}{xxxxxxxxxxx}

          \item[naFeatures]

          Numpy array of features,

          \item[fMin]

          Data frame containing the price information for all of the 
          stocks.

          \item[fMax]

          List of feature functions, most likely coming from features.py

          \item[bAbsolute]

          If true, min value will be scaled to fMin, max to fMax, if false,
          +-1 standard deviations will be scaled to fit between fMin and 
          fMax, i.e. {\textasciitilde}69\% of the values

          \item[bIgnoreLast]

          If true, last column is ignored (assumed to be classification)

        \end{Ventry}

      \end{quote}

      \textbf{Return Value}
    \vspace{-1ex}

      \begin{quote}
      list of (weights, shifts) to be used to normalize the query points

      \end{quote}

    \end{boxedminipage}

    \label{QSTK:qstkfeat:featutil:normQuery}
    \index{QSTK \textit{(package)}!QSTK.qstkfeat \textit{(package)}!QSTK.qstkfeat.featutil \textit{(module)}!QSTK.qstkfeat.featutil.normQuery \textit{(function)}}

    \vspace{0.5ex}

\hspace{.8\funcindent}\begin{boxedminipage}{\funcwidth}

    \raggedright \textbf{normQuery}(\textit{naQueries}, \textit{ltWeightShift})

    \vspace{-1.5ex}

    \rule{\textwidth}{0.5\fboxrule}
\setlength{\parskip}{2ex}
\setlength{\parskip}{1ex}
      \textbf{Parameters}
      \vspace{-1ex}

      \begin{quote}
        \begin{Ventry}{xxxxxxxxxxxxx}

          \item[naQueries]

          Numpy array of queries

          \item[ltWeightShift]

          List of weights and shift amounts to be applied to each query.

        \end{Ventry}

      \end{quote}

      \textbf{Return Value}
    \vspace{-1ex}

      \begin{quote}
      None, modifies naQueries

      \end{quote}

    \end{boxedminipage}

    \label{QSTK:qstkfeat:featutil:createKnnLearner}
    \index{QSTK \textit{(package)}!QSTK.qstkfeat \textit{(package)}!QSTK.qstkfeat.featutil \textit{(module)}!QSTK.qstkfeat.featutil.createKnnLearner \textit{(function)}}

    \vspace{0.5ex}

\hspace{.8\funcindent}\begin{boxedminipage}{\funcwidth}

    \raggedright \textbf{createKnnLearner}(\textit{naFeatures}, \textit{lKnn}={\tt 30}, \textit{leafsize}={\tt 10})

    \vspace{-1.5ex}

    \rule{\textwidth}{0.5\fboxrule}
\setlength{\parskip}{2ex}
\setlength{\parskip}{1ex}
      \textbf{Parameters}
      \vspace{-1ex}

      \begin{quote}
        \begin{Ventry}{xxxxxxxxxxx}

          \item[naFeatures]

          Numpy array of features,

          \item[fMin]

          Data frame containing the price information for all of the 
          stocks.

          \item[fMax]

          List of feature functions, most likely coming from features.py

          \item[bAbsolute]

          If true, min value will be scaled to fMin, max to fMax, if false,
          +-1 standard deviations will be scaled to fit between fMin and 
          fMax, i.e. {\textasciitilde}69\% of the values

          \item[bIgnoreLast]

          If true, last column is ignored (assumed to be classification)

        \end{Ventry}

      \end{quote}

      \textbf{Return Value}
    \vspace{-1ex}

      \begin{quote}
      None, data is modified in place

      \end{quote}

    \end{boxedminipage}

    \label{QSTK:qstkfeat:featutil:log500}
    \index{QSTK \textit{(package)}!QSTK.qstkfeat \textit{(package)}!QSTK.qstkfeat.featutil \textit{(module)}!QSTK.qstkfeat.featutil.log500 \textit{(function)}}

    \vspace{0.5ex}

\hspace{.8\funcindent}\begin{boxedminipage}{\funcwidth}

    \raggedright \textbf{log500}(\textit{sLog})

    \vspace{-1.5ex}

    \rule{\textwidth}{0.5\fboxrule}
\setlength{\parskip}{2ex}
\setlength{\parskip}{1ex}
      \textbf{Parameters}
      \vspace{-1ex}

      \begin{quote}
        \begin{Ventry}{xxxx}

          \item[sLog]

          Filename of features.

        \end{Ventry}

      \end{quote}

      \textbf{Return Value}
    \vspace{-1ex}

      \begin{quote}
      Nothing, logs features to desired location

      \end{quote}

    \end{boxedminipage}

    \label{QSTK:qstkfeat:featutil:getFeatureFuncs}
    \index{QSTK \textit{(package)}!QSTK.qstkfeat \textit{(package)}!QSTK.qstkfeat.featutil \textit{(module)}!QSTK.qstkfeat.featutil.getFeatureFuncs \textit{(function)}}

    \vspace{0.5ex}

\hspace{.8\funcindent}\begin{boxedminipage}{\funcwidth}

    \raggedright \textbf{getFeatureFuncs}()

    \vspace{-1.5ex}

    \rule{\textwidth}{0.5\fboxrule}
\setlength{\parskip}{2ex}
\setlength{\parskip}{1ex}
      \textbf{Return Value}
    \vspace{-1ex}

      \begin{quote}
      Tuple containing (list of functions, list of arguments, list of 
      names)

      \end{quote}

    \end{boxedminipage}

    \label{QSTK:qstkfeat:featutil:testFeature}
    \index{QSTK \textit{(package)}!QSTK.qstkfeat \textit{(package)}!QSTK.qstkfeat.featutil \textit{(module)}!QSTK.qstkfeat.featutil.testFeature \textit{(function)}}

    \vspace{0.5ex}

\hspace{.8\funcindent}\begin{boxedminipage}{\funcwidth}

    \raggedright \textbf{testFeature}(\textit{fcFeature}, \textit{dArgs})

    \vspace{-1.5ex}

    \rule{\textwidth}{0.5\fboxrule}
\setlength{\parskip}{2ex}
\setlength{\parskip}{1ex}
      \textbf{Parameters}
      \vspace{-1ex}

      \begin{quote}
        \begin{Ventry}{xxxxxxxxx}

          \item[fcFeature]

          Feature function to test

          \item[dArgs]

          Arguments to pass into feature function

        \end{Ventry}

      \end{quote}

      \textbf{Return Value}
    \vspace{-1ex}

      \begin{quote}
      Void

      \end{quote}

    \end{boxedminipage}


%%%%%%%%%%%%%%%%%%%%%%%%%%%%%%%%%%%%%%%%%%%%%%%%%%%%%%%%%%%%%%%%%%%%%%%%%%%
%%                               Variables                               %%
%%%%%%%%%%%%%%%%%%%%%%%%%%%%%%%%%%%%%%%%%%%%%%%%%%%%%%%%%%%%%%%%%%%%%%%%%%%

  \subsection{Variables}

    \vspace{-1cm}
\hspace{\varindent}\begin{longtable}{|p{\varnamewidth}|p{\vardescrwidth}|l}
\cline{1-2}
\cline{1-2} \centering \textbf{Name} & \centering \textbf{Description}& \\
\cline{1-2}
\endhead\cline{1-2}\multicolumn{3}{r}{\small\textit{continued on next page}}\\\endfoot\cline{1-2}
\endlastfoot\raggedright \_\-\_\-p\-a\-c\-k\-a\-g\-e\-\_\-\_\- & \raggedright \textbf{Value:} 
{\tt \texttt{'}\texttt{QSTK.qstkfeat}\texttt{'}}&\\
\cline{1-2}
\end{longtable}

    \index{QSTK \textit{(package)}!QSTK.qstkfeat \textit{(package)}!QSTK.qstkfeat.featutil \textit{(module)}|)}
